\documentclass{article}
\usepackage{algorithmic}
\usepackage[linesnumbered,lined,boxed,commentsnumbered]{algorithm2e}
\usepackage{outlines}

\begin{document}

\section{Questions}
\begin{outline}
  \1 you told us to prove that we can verify if a threshold $T$ will
  produce a valid solution. I'm not sure where to start with that, but I have
  a few ideas.
  \1 property 2 on next slide: does this enable our full solution? Or is
  this the property that just enables us to write the part of the algorithm
  that checks if we can fit everything within intervals of size $T$?
    \2 answer: 
    \2 if not, what property do we use to prove the above?
    \2 if not, what is property 2 useful for?
  \1 how do I write pseudocode
  \1 I assume once we prove that, we can just prove that checking every 
  $T$ from $0..\sum_{j \in 1..n} p_j$ is correct because we are enumerating
  all possibilities?
  \1 and how do we prove binomial search works? I have a binomial search
  written but since it uses integer division, I am not certain it covers
  everything. 
  \1 do we need to prove that the max $T$ could be is $\sum_{j \in 1..n}
  p_j$?
  \1 do we need to prove that the minimum $T$ could be is $\frac{\sum_{j
  \in 1..n} p_j}{m}$?
\end{outline}

\section{Problem Definition}
\begin{itemize}
  \item Input: number of processors $m$, number of tasks $n$, and chain of
        $n$ tasks $P = (m, n, [p_1, p_2, ..., p_n])$
  \item Output: $m$ disjoint intervals  $S = [I_1, I_2, ..., I_m]$ which
        partitions $[[1..n]]$. This can also be written as 
        \\$S^* = [[b_1,e_1[, [b_2,e_2[, ... , [b_m,e_m[]$ 
        \\where $b_1 = 1$, $b_i = e_{i-1}$, and $e_m = n$
  \item Metric: minimize the cost of the longest interval, e.g.
        $\min(\max_{i\in\{1..m\}}(\sum_{j \in{I_i}} p_j))$
\end{itemize}

\section{Properties}
Strong dominance property: Suppose problem $P$ has an optimal solution $S$
with value $V$. Each interval $I_i | i \in [1..m]$ will have value less than
$V$. (true but not useful...)

Weak dominance property: There exists an optimal solution $S$ to problem $P$
with associated $T$ such that
\\$\frac{\sum_{j \in 1..n} p_j}{m} < T \leq \sum_{j \in 1..n} p_j$ 
Where every interval except the last cannot take another task without
exceeding $T$. (why is this useful!)

More formal way to write this follows

$T^* = $ value of optimal solution
\\$T \geq T^*$

Given a threshold $T \geq T*$ where $T*$ is the value of the optimal
solution, $\exists S^* = ([b_1,e_1[, [b_2,e_2[, ... , [b_m,e_m[)$ which
partitions our tasks where 
\begin{itemize}
  \item $\exists k < m$
  \item $\forall i \leq k$, $\sum_{b_i \leq j < e_i} p_j \leq T$
  \item $\forall i \leq k$, $\sum_{b_i \leq j \leq e_i} p_j > T$
  \item $\forall i > k + 1$, $\sum_{b_i \leq j < e_i} p_j = 0$
\end{itemize}

In english: every problem has a solution where:
\begin{itemize}
  \item the first $k - 1$ intervals are saturated (unable to add the first
        thing from the next interval without exceeding $T$)
  \item every interval after $k$ is empty. (we may or may not use all 
        processors)
\end{itemize}
e.g. the load is balanced

This property enables CHECKING a given T because it if we prove it then we
know it is valid for all $T > T*$ where $T*$ is the optimal

\section{proof}
proof by transformation

start with optimal instance that does not have this structure and transform
it inot something that does

in this case, optimal means optimal for the decision problem of whether or
not we can fit everything in a threshold. So an optimal solution will always
be true

\section{Algorithm}
Check if for a given T we can build a correct solution with all interval
values less than or equal to T:

call this PROBE


Procedure: $probe$

\begin{algorithm}
  \KwIn{$n > 0$, $m > 0$, $T > 0$, $P = [p_1, p_2, ... p_n]$}
  \KwOut{returns true iff $T$ is a valid threshold}
  $last \leftarrow 0$\;
  $I \leftarrow$ empty list of intervals\;
  \For{$i=1 \rightarrow m$}{
    find $j \ni \sum_{last \leq k < j} p_k \leq T$ and 
    $\sum_{last \leq k \leq j} p_k > T$ $I[i] 
    \leftarrow [last,j[$\;
    $last \leftarrow j$\;
  }
  \If{$last = n$}{
    \Return{true}\;
  } 
  \Else{
    \Return{false}\;
  }
\end{algorithm}

Once we have $probe$, we can just check every $T$ where 
$\frac{\sum_{j \in 1..n} p_j}{m} \leq T \leq \sum_{j \in 1..n} p_j$
\\Procedure: $findT$
\begin{algorithmic}
  \REQUIRE $n > 0$, $m > 0$, $P = [p_1, p_2, ... p_n]$
  \ENSURE $T^*$ is minimal
  \STATE $T^* \leftarrow \infty$
  \FOR{$T = \frac{\sum_{j \in 1..n} p_j}{m} \rightarrow \sum_{j \in 1..n} p_j$}
    \IF{$probe$(T) \AND $T < T^*$}
      \STATE $T^* \leftarrow T$
    \ENDIF
  \ENDFOR
  \RETURN{$T^*$}
\end{algorithmic}

Once you find the optimal $T$, building the solution is simply $probe$, with
the return adjusted to give an interval instead of a boolean
\\Procedure: $buildSolution$
\begin{algorithmic}
  \REQUIRE $n > 0$, $m > 0$, $T = T^*$, $P = [p_1, p_2, ... p_n]$
  \ENSURE $I$ is an optimal partitioning of $P$
  \STATE $last \leftarrow 0$
  \STATE $I \leftarrow$ empty list of intervals
  \FOR{$i=1 \rightarrow m$}

    \STATE{find $j \ni \sum_{last \leq k < j} p_k \leq T$ and
    $\sum_{last \leq k \leq j} p_k > T$}
    \STATE{$I[i] \leftarrow [last,j[$}
    \STATE{$last \leftarrow j$}

  \ENDFOR
  \RETURN{$I$}
\end{algorithmic}

\section{Optimizations}
  One possible optimization is using binary search so we probe fewer things

  Procedure: $findTbinarySearch$
  \begin{algorithmic}
    \REQUIRE $n > 0$, $m > 0$, $P = [p_1, p_2, ... p_n]$
    \ENSURE $T^*$ is minimal
    \STATE $T^* \leftarrow \infty$
    \FOR{$T = binarySearch(\frac{\sum_{j \in 1..n} p_j}{m} , \sum_{j \in 1..n} p_j$)}
      \IF{$probe$(T) \AND $T < T^*$}
        \STATE $T^* \leftarrow T$
      \ENDIF
    \ENDFOR
    \RETURN{$T^*$}
  \end{algorithmic}

  $n^2$ potential intervals: compute them all and sort them, then binary search
  on those intervals
  
  this is somehow LESS expensive than $log(m-1/m sum p_i)$


\end{document}
