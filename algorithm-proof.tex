\documentclass{article}
\author{Riley Weber}
\title{Chain on Chain}

\begin{document}
\maketitle

\section{Problem Description}

\begin{tabbing}
	\bfseries left \quad \quad \quad \=\bfseries paragraph \kill
	\textbf{Input:}  \> chain of $n$ tasks $P = [p_1, p_2, ..., p_n]$
	\\
	\textbf{Output:} \> $m$ disjoint intervals  $S = [I_1, I_2, ..., I_m]$ which
	partitions $[[1..n]]$ 
	\\
	\textbf{Metric:} \> minimize the cost of the longest interval e.g. 
	\\
	\>$\min(\max_{i\in\{1..m\}}(\sum_{j \in{I_i}} p_j))$ 
	% \\note: we define the function $value (S)$ to denote the value of the metric for $S$
\end{tabbing}

\section{Definitions}
\textbf{Optimal:} having the lowest possible value. Because our metric uses the
$max$ operation, there is more than one solution that is considered "optimal"
under this definition.

\section{Property}
\begin{itemize}
	% this property is incorrect! See comment starting line 69
\item Suppose there is a problem $P = [p_1, p_2, ..., p_n]$. There exists an
optimal solution $S = [I_1, I_2, ... I_m]$ for problem $P$ 
where $S' = [I_1, I_2, ... I_{m-1}]$ is an optimal solution for 
$P' = [[p_1, p_2, ..., p_n] \setminus I_m]$
\item THERE IS AN optimal. It is not necessary that an optimal solution have
this property, but there exists an optimal solution that does have this
property.
\item prove the built solution won't be bigger. It will be equal to or
smaller in value
\end{itemize}

\section{Proof}
\begin{enumerate}
	\item Suppose there exists a solution 
	$S = [I_1, I_2, ..., I_m]$ that is optimal for problem 
	$P = [p_1, p_2, ..., p_n]$, and that $value(S)$ is denoted $V$
	\item Suppose there exists a solution $S' = [I_1, I_2, ..., I_{m-1}]$ for
	problem $P'= [[p_1, p_2, ..., p_n] \setminus I_m]$, where $I_m$ is an
	interval at the beginning or end of the chain. The valule of $S'$ is
	$V' = \max_{i\in\{1..m-1\}}(\sum_{j \in{I_i}} p_j))$ 
	Suppose that this solution is not optimal.
	\item Because $S'$ is not optimal for $P'$, there exists a solution $S^* =
	[I_1*, I_2*, ..., I_{m-1}*]$ for problem $P'$ with value $V^*$, and $V^* <
	V'$ \\(remember, the goal is to minimize our metric) 

	% prove that this new solution is valid
	\item By adding $I_m$ to solution $S^*$, we can build a solution to our
	original problem $P$. Let us call this solution $S^{**} = [I_1*, I_2*, ...,
	I_{m-1}*, I_m]$, and its value $V^{**} = \max_{i\in\{1..m\}}(\sum_{j
	\in{I_i}} p_j))$.

	% are you sure??? This is not necessarily true what if there are 4
	% processors, 3 tasks of size 1, and one task of size 10? Each interval
	% just contains one task, and I_m contnains the final task of size 10? Once
	% I_m is removed, even if you put all 3 tasks on one processor (a
	% sub-optimal solution), the final solution will still be optimal. 
	\item Because $V^{*} < V'$, it must also be true that $V^{**} < V$

	\item However, this is impossible, because $S$ is optimal. 
	\item Therefore, $S'$ must also be optimal and our property is valid
\end{enumerate}

\end{document}
     
