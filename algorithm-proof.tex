\documentclass[landscape]{slides}
\usepackage[landscape]{geometry}

\usepackage{xcolor}
\usepackage{pagecolor}
\usepackage{lipsum}  
\usepackage{enumitem} % to remove separation between list items
% \usepackage{setspace} % finer control over line spacing

\definecolor{darkgray}{HTML}{212121}
\pagecolor{darkgray}
\color{lightgray}
\setlist[itemize]{noitemsep}

\begin{document}
% \setstretch{1.1}

\begin{slide} % PROBLEM STATEMENT
	\begin{itemize}
    \item Input: chain of $n$ tasks $P = [p_1, p_2, ..., p_n]$
    \item Output: $m$ disjoint intervals  $S = [I_1, I_2, ..., I_m]$ which
    			partitions $[[1..n]]$
    \item Metric: minimize the cost of the longest interval, e.g.
					$\min(\max_{i\in\{1..m\}}(\sum_{j \in{I_i}} p_j))$
					\\note: we define the function $value (S)$ to denote the value of 
					the metric for $S$
	\end{itemize}
\end{slide}

\begin{slide} % PROPERTY
	\begin{itemize}
		% this property is incorrect! See comment starting line 69
	\item If $S = [I_1, I_2, ... I_m]$ is an optimal solution for problem 
	$P = [p_1, p_2, ..., p_n]$, then $S' = [I_1, I_2, ... I_{m-1}]$ is an optimal
	solution for $P' = [[p_1, p_2, ..., p_n] \setminus I_m]$
	\item Note: $P'$ is $P$ with everything associated with $I_m$ removed
	\end{itemize}
\end{slide}

\begin{slide} % PROPERTY
	\begin{itemize}
		% PROBABLY WRONG
		\item We define "optimal" to mean having the lowest possible value. Because
		our metric uses the $max$ operation, there is more than one solution that
		is considered "optimal" under this definition.
	\end{itemize}
\end{slide}

\begin{slide} % PROOF 1
	\begin{enumerate}
		\item Suppose there exists a solution 
		$S = [I_1, I_2, ..., I_m]$ that is optimal for problem 
		$P = [p_1, p_2, ..., p_n]$, and that $value(S)$ is denoted $V$
		\item Suppose there exists a solution $S' = [I_1, I_2, ..., I_{m-1}]$ for
		problem $P'= [[p_1, p_2, ..., p_n] \setminus I_m]$, where $I_m$ is an
		interval at the beginning or end of the chain. The valule of $S'$ is
		$V' = \max_{i\in\{1..m-1\}}(\sum_{j \in{I_i}} p_j))$ 
		Suppose that this solution is not optimal.
		\item Because $S'$ is not optimal for $P'$, there exists a solution $S^* =
		[I_1*, I_2*, ..., I_{m-1}*]$ for problem $P'$ with value $V^*$, and $V^* <
		V'$ \\(remember, the goal is to minimize our metric) 

		% prove that this new solution is valid
		\item By adding $I_m$ to solution $S^*$, we can build a solution to our
		original problem $P$. Let us call this solution $S^{**} = [I_1*, I_2*, ...,
		I_{m-1}*, I_m]$, and its value $V^{**} = \max_{i\in\{1..m\}}(\sum_{j
		\in{I_i}} p_j))$.

		% are you sure??? This is not necessarily true what if there are 4
		% processors, 3 tasks of size 1, and one task of size 10? Each interval
		% just contains one task, and I_m contnains the final task of size 10? Once
		% I_m is removed, even if you put all 3 tasks on one processor (a
		% sub-optimal solution), the final solution will still be optimal. 
		\item Because $V^{*} < V'$, it must also be true that $V^{**} < V$

		\item However, this is impossible, because $S$ is optimal. 
		\item Therefore, $S'$ must also be optimal and our property is valid
	\end{enumerate}
\end{slide}

\end{document}
     