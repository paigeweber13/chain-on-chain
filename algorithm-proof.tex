\documentclass[landscape]{slides}
\usepackage[landscape]{geometry}

\usepackage{xcolor}
\usepackage{pagecolor}
\usepackage{lipsum}  
\usepackage{enumitem} % to remove separation between list items
% \usepackage{setspace} % finer control over line spacing

\definecolor{darkgray}{HTML}{212121}
\pagecolor{darkgray}
\color{lightgray}
\setlist[itemize]{noitemsep}

\begin{document}
% \setstretch{1.1}

\begin{slide} % PROBLEM STATEMENT
	\begin{itemize}
    \item Input: chain of $n$ tasks $P = [p_1, p_2, ..., p_n]$
    \item Output: $m$ disjoint intervals  $S = [I_1, I_2, ..., I_m]$ which
    			partitions $[[1..n]]$
    \item Metric: minimize the cost of the longest interval, e.g.
					$\min(\max_{i\in\{1..m\}}(\sum_{j \in{I_i}} p_j))$
					\\note: we define the function $value (S)$ to denote the value of 
					the metric for $S$
	\end{itemize}
\end{slide}

\begin{slide} % PROPERTY
	\begin{itemize}
	\item If $S = [I_1, I_2, ... I_m]$ is an optimal solution for problem 
	$P = [p_1, p_2, ..., p_n]$, then $S' = [I_1, I_2, ... I_{m-1}]$ is an optimal
	solution for $P' = [[p_1, p_2, ..., p_n] \setminus I_m]$
	\item Note: $P'$ is $P$ with everything associated with $I_m$ removed
	\end{itemize}
\end{slide}

\begin{slide} % PROOF 1
	\begin{itemize}
		\item Suppose there exists a solution 
		$S = [I_1, I_2, ..., I_m]$ that is optimal for problem 
		$P = [p_1, p_2, ..., p_n]$, and that $value(S)$ is denoted $V$
		\item Suppose there exists a solution $S' = [I_1, I_2, ..., I_{m-1}]$
		for problem $P'= [[p_1, p_2, ..., p_n] \setminus I_m]$, but that this
		solution is not optimal.
		\item The value of $P'$ is designated $V'$. Therefore, 
		\\$V' = V - value(I_m)$
		\item Because $S'$ is not optimal for $P'$, there exists a solution $S^*$
		for problem $P'$ with value $V^*$, and $V^* < V'$ \\(remember, the goal is
		to minimize our metric) 
	\end{itemize}
\end{slide}

\begin{slide} % PROOF 2
	\begin{itemize}
		\item By adding $I_m$ to solution $S^*$, we obtain a solution to our
		original problem $P$. Let us call this solution $S^{**}$, and its value
		$V^{**}$
		\item Therefore, the value of $V^{**}$ is as follows: 
		\\$V^{**} = V^* + value(I_m)$
		\item this may be rewritten as $V^* = V^{**} - value(I_m)$
	\end{itemize}
\end{slide}

\begin{slide} % PROOF 2
	\begin{itemize}
		\item recall $V^* = V^{**} - value(I_m)$
		\item recall $V' = V - value(I_m)$ 
		\item recall $V^* < V'$
		\item by substitution: $V^{**} - value(I_m) < V - value(I_m)$
		\item and elimination: $V^{**} < V$
		\item this is impossible, because $S$ is optimal. 
		\item therefore, $S'$ must also be optimal and our property is valid
	\end{itemize}
\end{slide}

\begin{slide} % PROOF 2
	\begin{itemize}
		\item in other words, because $V^*$ is less than $V'$, then $V^{**}$ must
		be less than $V$
		\item However, this is impossible, because we began the proof by declaring
		$S$ to be optimal. This implies that $V$ is the lowest possible value for
		any solution to $P$
		\item Therefore, if $S$ is optimal, then $S'$ must also be optimal and our
		property is true.
	\end{itemize}
\end{slide}

\end{document}
     